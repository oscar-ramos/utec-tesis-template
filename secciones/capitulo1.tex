\chapter{TÍTULO DEL CAPÍTULO}

Inicie aquí el texto, utilizando sangría de 1.25 cm. en el primer
párrafo. Continúe el segundo párrafo.

Continúe el segundo párrafo

\section{Primer subtítulo}

En la actualidad, las unidades de información hacen frente a muchos cambios
debido a los avances tecnológicos, explosión informativa, nuevos recursos y
soportes, por lo cual se implementan servicios innovadores que les permitan a
sus usuarios tener acceso a muchas fuentes de información. Para asegurar el
acceso y uso de los servicios los usuarios requieren poseer una serie de
habilidades que les permitan identificar, recuperar, manejar, discernir,
organizar, utilizar y comunicar la información de manera eficaz para la toma de
decisiones \cite{citacion1}.

Entre las causas se pueden considerar la falta de tiempo asignado al taller,
las limitaciones en cuanto a laboratorios, el poco personal, la falta de una
correcta selección de los contenidos, así como, una calificación que asegure el
cumplimiento de los objetivos.


\section{Segundo subtítulo}

\subsection{División del segundo subtítulo}

En la actualidad, las unidades de información hacen frente a muchos cambios
debido a los avances tecnológicos, explosión informativa, nuevos recursos y
soportes, por lo cual se implementan servicios innovadores que les permitan a
sus usuarios tener acceso a muchas fuentes de información. Para asegurar el
acceso y uso de los servicios los usuarios requieren poseer una serie de
habilidades que les permitan identificar, recuperar, manejar, discernir,
organizar, utilizar y comunicar la información de manera eficaz para la toma de
decisiones.

\begin{equation}
  \label{eq:1}
  a+b=\sqrt{\frac{4}{3}}
\end{equation}

Entre las causas se pueden considerar la falta de tiempo asignado al taller,
las limitaciones en cuanto a laboratorios, el poco personal, la falta de una
correcta selección de los contenidos, así como, una calificación que asegure el
cumplimiento de los objetivos.

\begin{figure}%[h]
  \centering
  \includegraphics{images/testfig.png}
  \includegraphics{images/testfig.png}
  \captionsetup{font=footnotesize}
  \caption{Scheme showing the architecture of a generic kinematic task.}
  \label{fig:diagram2}
\end{figure}



\section{Tercer subtítulo}

\subsection{División del tercer subtítulo}

En la actualidad, las unidades de información hacen frente a muchos cambios
debido a los avances tecnológicos, explosión informativa, nuevos recursos y
soportes, por lo cual se implementan servicios innovadores que les permitan a
sus usuarios tener acceso a muchas fuentes de información. Para asegurar el
acceso y uso de los servicios los usuarios requieren poseer una serie de
habilidades que les permitan identificar, recuperar, manejar, discernir,
organizar, utilizar y comunicar la información de manera eficaz para la toma de
decisiones.

Entre las causas se pueden considerar la falta de tiempo asignado al taller,
las limitaciones en cuanto a laboratorios, el poco personal, la falta de una
correcta selección de los contenidos, así como, una calificación que asegure el
cumplimiento de los objetivos.

